\documentclass{beamer}

\mode<presentation> {
\usetheme{Madrid}
}

\usepackage{graphicx}
\usepackage{booktabs}
\usepackage[utf8x]{inputenc}

\usepackage[T1]{fontenc}
\usepackage[autostyle=true,german=quotes]{csquotes}
\usepackage{listings}
\usepackage{amssymb}
\usepackage{amsmath}
\usepackage{amsthm}
\usepackage{amsfonts}

\usepackage{tikz}
\usepackage{gensymb}
\usepackage{stmaryrd}

\usepackage{pgfplots}
\usetikzlibrary{backgrounds}
\usetikzlibrary{patterns}

%----------------------------------------------------------------------------------------
%   TITLE PAGE
%----------------------------------------------------------------------------------------

\title[Pollard-Rho]{Zufällige Abbildungen und Faktorisieren via Pollard-Rho}

\author{Felix Potthast}
\date{02.03.2016}

\begin{document}

\begin{frame}
	\frametitle{Analyse von Algorithmen}
	\titlepage
\end{frame}

%----------------------------------------------------------------------------------------
%   CONTENTS
%----------------------------------------------------------------------------------------

\begin{frame}
	\frametitle{Gliederung}
	\tableofcontents
\end{frame}

%----------------------------------------------------------------------------------------
%   PRESENTATION SLIDES
%----------------------------------------------------------------------------------------

\section{Zufällige Abbildungen}

%------------------------------------------------

\begin{frame}
	\frametitle{Pseudozufallszahlen}
	\begin{itemize}
		\item Arithmetische, deterministische Methoden zur Generierung zufällig wirkender Folgen
		\item Verwendung z.B. für Monte-Carlo Verfahren oder Kryptographie
		\item Übliche Form: \(x_{i+1}=f(x_i)\) mit Seed \(x_0\), \(x_i \in \{0,\cdots,m-1\} = \mathbb Z/m\mathbb Z\)
		\item Gewünschte Eigenschaften: lange Periode, Gleichverteilung
	\end{itemize}
\end{frame}

%------------------------------------------------

\begin{frame}
\frametitle{Periodische Folgen}
	\begin{block}{Definition: Periodische Folge}
		Sei \((x_i)_{i \geq 0}\) eine Folge.  \(x_i\) ist periodisch mit der Periodenlänge \(\lambda\)
		und der Vorperiodenlänge \(\mu\) (und \(\rho=\mu+\lambda\)), wenn gilt:\\
		\(x_0,x_1,\;\dots\;, x_\mu, \;\dots\;, x_{\mu + \lambda - 1}\) sind paarweise verschieden,
		für \(n \geq \mu\) gilt aber: \(x_{n+\lambda}=x_n\)
	\end{block}
	Sei \(f: \mathbb Z/m\mathbb Z \to \mathbb Z/m\mathbb Z \) beliebig. \\
	Dann ist die Folge \((x_i)_{i \geq 0}\), \(x_0 \in \mathbb Z/m\mathbb Z\), \(x_{i+1}=f(x_i)\)
	periodisch mit \(1 \leq \lambda \leq m\) und \(0 \leq \mu\) sowie \(\mu+\lambda=\rho \leq m\).
\end{frame}

%------------------------------------------------

\begin{frame}
	\frametitle{Zufällige Abbildungen}
	Es gibt \(m^m\) Abbildungen in obiger Form.
	Sei \(f\) nun eine daraus zufällig gewählte Abbildung.
	Dann ist die Wahrscheinlichkeit die gewünschte maximale Periodenlänge \(\lambda=m\) zu erhalten gegeben durch: \(\frac{(m-1)!}{m^m}\)
	(Erläuterung: \(m!\) ist die Anzahl der Permutationen, der Zyklus bleibt aber gleich für \(m\) Verschiebungen) \\
	Eine Approximation der Fakultät mit der Stirlingformel ergibt dann:
	\[
		\frac{(m-1)!}{m^m} = \frac{m!}{m^{(m+1)}} \approx \frac{\sqrt{2 \pi m} \left(\frac{m}{e}\right)^m}{m^{(m+1)}} = \sqrt{\frac{2 \pi}{m}} \cdot \left(\frac{1}e\right)^m
	\]
\end{frame}

\begin{frame}
	\frametitle{Zufällige Abbildungen}
	Die Wahrscheinlichkeit, dass eine zufällig gewählte Abbildung \(f\) einen Zyklus der Länge 1 hat, ist gegeben durch:
	\[
		1-\left(\frac{m-1}{m}\right)^m = 1-\left(1-\frac{1}{m}\right)^m >  1 - \frac{1}{e} \approx 0.63
	\]
\end{frame}

%------------------------------------------------

\begin{frame}
	\frametitle{Zufällige Abbildungen}
	Sei \(f\) wieder eine zufällig gewählte Abbildung mit nun ebenfalls zufällig gewähltem Seed \(x_0\).
	Die Wahrscheinlichkeit, eine Folge mit Vorperiodenlänge \(\mu\) und Periodenlänge \(\lambda\) zu erhalten, ist:
	\[
		P_m(\mu,\lambda)=\frac{1}{m}\prod_{k=1}^{\mu + \lambda}\left(1-\frac{k}{m}\right)
	\]
	Erläuterung:
	Die ersten \(\mu + \lambda\) Werte müssen paarweise verschieden sein,
	danach muss ein bestimmter Wert erreicht werden, um \(\lambda\) zu erhalten.
	Der Durchschnitt für \(\mu + \lambda\) ist dann gegeben durch:
	\[
		\sum_{\lambda=1}^{m} \sum_{\mu=0}^{m-\lambda}  (\lambda+\mu) P_m(\mu,\lambda) = Q(m)
		\approx \sqrt{\frac{\pi m}{2}} \approx 1.253 \cdot \sqrt{m}
	\]
\end{frame}

%------------------------------------------------

\begin{frame}
	\frametitle{Zufällige Abbildungen}
	\(Q(m)\) ist die Ramanujan Q-Funktion gegeben durch 
	\[
		Q(m)= \sum_{k=1}^m \frac{m!}{(m-k)!m^k}
	\]
	Für \(k=o\left(m^\frac{2}{3}\right)\) und \(m \to \infty\) gilt:
	\[
		\frac{m!}{(m-k)!m^k}=e^\frac{-k^2}{2m} \left( 1 + O\left(\frac{k}{m}\right) + O\left(\frac{k^3}{m^2}\right) \right)
	\]
	Für alle \(k\), mit \(m \to \infty\) gilt:
	\[
		\frac{m!}{(m-k)!m^k}=e^\frac{-k^2}{2m} + O\left(\frac{1}{\sqrt{m}}\right)
	\]
\end{frame}

%------------------------------------------------

\begin{frame}
	\frametitle{Zufällige Abbildungen}
	Beweis:
	\begin{align*}
		\frac{m!}{(m-k)!m^k} = \frac{\prod\limits_{j=1}^{k-1}(m-j)}{m^k} = \prod_{j=1}^{k-1} \left(1 - \frac{j}{m}\right)
		&= e^{\ln{\left( \prod_{j=1}^{k-1} \left(1 - \frac{j}{m} \right) \right)}} \\
		&= e^{ \sum_{j=1}^{k-1} \ln{\left(1 - \frac{j}{m} \right)}}
	\end{align*}
	Da \(k=o(n)\) wenden wir die Taylor-Formel auf den Logarithmus an:
	\(\ln{(1+x)} = x + O(x^2)\) mit \(x=\frac{-j}{m}\)
	\begin{align*}
		e^{ \sum_{j=1}^{k-1} \ln{\left(1 - \frac{j}{m} \right)}}
		= e^{\sum_{j=1}^{k-1} \left(-\frac{j}{m} + O\left(\frac{j^2}{m^2}\right)\right)}
		= e^{\left(-\frac{k(k-1)}{2m} + O\left(\frac{k^3}{m^2}\right)\right)}
	\end{align*}
\end{frame}

%------------------------------------------------

\begin{frame}
	\frametitle{Zufällige Abbildungen}
	Dann können wir für den Fall \(k=o\left(m^{\frac{2}{3}}\right)\) die Taylor-Formel für die Exponentialfunktion anwenden:
	\(e^x = 1 + O(x) \)
	\begin{align*}
		e^{\left(-\frac{k(k-1)}{2m} + O\left(\frac{k^3}{m^2}\right)\right)}
		&= e^{\left(\frac{-k^2}{2m} + \frac{k}{2m} + O\left(\frac{k^3}{m^2}\right)\right)} \\
		= e^{\left(\frac{-k^2}{2m}\right)} \cdot e^{\left(\frac{k}{2m} + O\left(\frac{k^3}{m^2}\right)\right)}
		&= e^\frac{-k^2}{2m} \left( 1 + O\left(\frac{k}{m}\right) + O\left(\frac{k^3}{m^2}\right) \right)
	\end{align*}
	\(\qquad\) \hfill \(\blacksquare\) \\
\end{frame}

%------------------------------------------------

\begin{frame}
	\frametitle{Zufällige Abbildungen}
	Dann betrachten wir noch die Approximationsformel für alle k:
	Sei \(k_0 = m^{\frac{3}{5}} \),
	dann gilt für \(k \leq k_0\) die obige Approximationsformel und wir setzen \(x=\frac{k}{\sqrt{2m}}\):
	\begin{align*}
		&e^\frac{-k^2}{2m} \left( 1 + O\left(\frac{k}{m}\right) + O\left(\frac{k^3}{m^2}\right) \right)
		= e^\frac{-k^2}{2m} + e^{-x^2} O\left(\frac{k}{m}\right) +  e^{-x^2} O\left(\frac{k^3}{m^2}\right) \\
		= &e^\frac{-k^2}{2m} + e^{-x^2} O\left(\frac{x\sqrt{2m}}{m}\right) +  e^{-x^2} O\left(\frac{{\left( x \sqrt{2m}\right)}^3 }{m^2}\right) \\
		= &e^\frac{-k^2}{2m} + xe^{-x^2} O\left(\frac{1}{\sqrt{m}}\right) +  x^3e^{-x^2} O\left(\frac{1}{\sqrt{m}}\right)
	\end{align*}
	Es gilt: \(xe^{-x^2} = O(1)\) und \(x^3e^{-x^2} = O(1)\) und daher:
	\(
		e^\frac{-k^2}{2m} + O\left(\frac{1}{\sqrt{m}}\right)
	\)
\end{frame}

%------------------------------------------------

\begin{frame}
	\frametitle{Zufällige Abbildungen}
	Sei nun \(k \geq k_0\), dann gilt:
	\[
		\frac{m!}{(m-k_0)!m^{k_0}} \leq \frac{m!}{(m-k)!m^k}
	\]
	und da \(k_0=o\left(m^{\frac{2}{3}}\right)\):
	\begin{align*}
		&\frac{m!}{(m-k_0)!m^{k_0}} = e^\frac{-k_0^2}{2m} + O\left(\frac{1}{\sqrt{m}}\right)
		= e^\frac{-\sqrt[5]{m}}{2} + O\left(\frac{1}{\sqrt{m}}\right) \\
		\implies &\frac{m!}{(m-k)!m^k} = O\left(\frac{1}{\sqrt{m}}\right)
	\end{align*}
\end{frame}
%------------------------------------------------

\begin{frame}
	\frametitle{Zufällige Abbildungen}
	Daraus lässt sich jetzt folgende Asymptotische Formel für \(Q(m)\) herleiten:
	\[
		Q(m)= \sum_{k=1}^m \frac{m!}{(m-k)!m^k} = \sqrt{\frac{\pi m}{2}} + O(1)
	\]

	Wir benutzen für unterschiedliche Bereiche der Summe unterschiedliche Fehlerabschätzungen: sei \(k_0=o\left( m^{\frac{2}{3}}\right)\)
	\begin{align*}
		Q(m)&= \sum_{k=1}^m \frac{m!}{(m-k)!m^k}
		= \sum_{k=1}^{k_0} \frac{m!}{(m-k)!m^k}
		+ \sum_{k=k_0+1}^{m} \frac{m!}{(m-k)!m^k} \\	
		&= \sum_{k=1}^{k_0} e^{\frac{-k^2}{2m}} \left( 1 + O\left(\frac{k}{m}\right) + O\left(\frac{k^3}{m^2}\right) \right)+ \Delta
	\end{align*}
\end{frame}

%------------------------------------------------

\begin{frame}
	\frametitle{Zufällige Abbildungen}
	Wir betrachten dabei \(\Delta\) als vernächlässigbar, da exponentiell klein.
	Wir nehmen auch an, dass alle weiteren Terme vernachlässigbar sind und betrachten nun:
	\( \sum_{k=1}^{\infty} e^{\frac{-k^2}{2m}} + O(1)\).
	Nach der Euler-Maclaurin Formel gilt:
	\begin{align*}
		\sum_{i = m}^{n} f(i) &= \int_{m}^{n} f(x) \,dx 
		+ \frac{f(n) + f(m)}{2}\\
		&+ \sum_{j = 1}^{k} \frac{B_{2j}}{(2j)!} \left( f^{(2j-1)}(n) - f^{(2j-1)}(m) \right) + R_{2k}(m,n)
	\end{align*}
\end{frame}

%------------------------------------------------

\begin{frame}
	\frametitle{Zufällige Abbildungen}

	Daraus folgt dann:

	\[
		Q(m)= \sum_{k=1}^{\infty} e^{\frac{-k^2}{2m}} + O(1)
		= \int_0^{\infty} e^{-\frac{k^2}{2m}} dk + O(1)
	\]
	Dann substituieren wir \(k=\varphi(x)=\frac{x}{\sqrt{m}}\):
	\[
		\sqrt{m} \int_{0}^{\infty} e^{-\frac{\left(\frac{k}{\sqrt{m}}\right)^2}{2m} }\left(\frac{1}{\sqrt{m}}\right) dx + O(1)
		= \sqrt{m} \int_{\varphi(0)}^{\varphi(\infty)} e^{-\frac{x^2}{2}} dx + O(1)
	\]
	wobei
	\[
		\int_0^{\infty} e^{-\frac{x^2}{2}} dx = \sqrt{\frac{\pi}{2}}
	\]
	und daher insgesamt:
	\[
		Q(m) = \sqrt{\frac{\pi}{2m}} + O(1)
	\]
	\(\qquad\) \hfill \(\blacksquare\) \\
\end{frame}

%------------------------------------------------

\begin{frame}
	\frametitle{Zufällige Abbildungen}
	Die Wahrscheinlichkeit, in einem Zyklus der Länge \(\lambda=1\) zu enden, ist dann:
	\[
		\sum_{\mu=0}^{m-1} P_m(\mu,1) = \frac{1}{m} \sum_{k=1}^{m} \frac{m!}{(m-k)! m^k} \approx \frac{1}{m} \sqrt{\frac{\pi}{2 \cdot m}} \approx \frac{1.25}{m\sqrt{m}}
	\]
\end{frame}

%------------------------------------------------

\begin{frame}
	\frametitle{Zufällige Abbildungen: Fazit}
	\begin{itemize}
		\item Durchschnittliche Zyklenlänge relativ gering
		\item Für Zufallszahlengeneratoren sind hohe Zyklenlängen erwünscht
		\item Zufällige Zufallszahlengeneratoren sind daher nicht Optimal
	\end{itemize}
\end{frame}

%------------------------------------------------

\section{Zyklensuche}

%------------------------------------------------

\begin{frame}
	\frametitle{Zyklensuche nach Floyd}
	Man vergleiche jeweils \(x_i\) mit \(x_{2 i}\),
	es werden also der Reihe nach größere Werte für \(\mu\) und \(\lambda\) getestet.
	Wenn \(x_k = x_{2 k}\) gilt: \(\mu < i\) und \(\lambda\) ist ein Vielfaches von \(i\).
	Dann kann man in einem weiteren Schritt die exakten Werte herausfinden.
	Der Algorithums braucht mindestens \(\mu + 1\) und maximal \(\mu + \lambda\) Schritte um einen Zyklus zu finden.
	Um die exakten Werte von \(\mu\) und \(\lambda\) zu finden werden nochmals \(\mu + \lambda\) Schritte gebraucht.
	In jedem Schritt wird \(f\) einmal ausgewertet und es findet ein Vergleich statt.
	
\end{frame}

%------------------------------------------------

\begin{frame}
	\frametitle{Zyklensuche nach Brent}
	
	Man vergleiche jeweils \(x_i\) mit \(x_{\ell(i)-1}\), wobei
	\begin{block}{Definition:}
		\[
			\ell(n) := max \{ 2^x \;|\; 2^x \leq n,\; x\in \mathbb N_0\} = 2^{\lfloor \operatorname{log_2}(n) \rfloor}
		\]
	\end{block}
	Hier testet man also für \(\mu\) exponentiell ansteigende Werte, für jedes \(\mu\) welches getestet wird
	probiert man dann alle \(\lambda\) von \(1\) bis \(\mu + 1\) aus, woraufhin man das \(\mu\) weiter erhöht.
	Die Zyklensuche nach Brent braucht nie mehr Schritte als die Zyklensuche nach Floyd und man erhält sofort die exakte Periodenlänge.
	\(\mu\) muss auch hier in einem weiteren Schritt erlangt werden.
	Auch dieser Algorithums braucht mindestens \(\mu + 1\) und maximal \(\mu + \lambda\) Schritte, um einen Zyklus zu finden.
	Werden die exakten Werte für \(\mu\) und \(\lambda\) gebraucht, werden weitere \( \mu \) Schritte benötigt.
\end{frame}

%------------------------------------------------

\section{Pollard-Rho Faktorisierung}

%------------------------------------------------

\begin{frame}
	\frametitle{Grundbegriffe}
	\begin{block}{Definition: Teiler}
		\[
			a \mid b \iff \exists x \in \mathbb Z: a \cdot x = b
		\]
	\end{block}
	\begin{block}{Definition: Primzahl}
		\[
			p \text{ ist Primzahl} \iff p \in \mathbb P \iff p>1 \wedge (a \mid p \implies a=1 \vee a=p) \
		\]
	\end{block}
	\begin{block}{Definition: Kongruenz}
		\[
			a \equiv b \pmod m \iff m \mid (a-b)
		\]
	\end{block}
	\begin{block}{Definition: Modulo-Abbildung}
		\[
			\bmod : \mathbb Z \times \mathbb N \to \mathbb Z,
			\quad (a,m) \mapsto a\;\bmod\;m:= a - \left \lfloor \frac{a}{m} \right \rfloor \cdot m
		\]
	\end{block}
\end{frame}

%------------------------------------------------

\begin{frame}
	\frametitle{Grundbegriffe}
	\begin{block}{Größter gemeinsamer Teiler}
		\[
			\operatorname{ggT}(a, b) = \max \{d \mid (d \mid a) \wedge (d \mid b) \}
		\]
	\end{block}
	\begin{block}{Primfaktorzerlegung}
		Für jede natürliche Zahl \(n \in \mathbb N \) gibt es Primzahlen \(p_k < p_{k+1}\) und Exponenten \(\alpha_k\), so dass:
		\[
			n=\prod_{k=1}^M p_k^{\alpha_k}
		\]
	\end{block}

\end{frame}

%------------------------------------------------

\begin{frame}
	\frametitle{Faktorisierung nach Pollard-Rho}
	Sei \(f(x)\) ein Polynom mit ganzzahligen Koeffizienten und \(n\) die Zahl, die zu faktorisieren ist.\\
	Dann gilt \( x_i = x_j \bmod n \implies f(x_i) = f(x_j)\bmod n\) \\
	Nun betrachten wir die Folgen \((x_{a_{i+1}})_{i \geq 0}: x_{i+1}=f(x_i) \bmod a\) und \((x_{n_{i+1}})_{i \geq 0}: x_{i+1}=f(x_i) \bmod n\)
	als zwei zufällige periodische Folgen, wobei \(\mu_a\), \(\mu_n\) und \(\lambda_a\), \(\lambda_n\)
	die Vorperiodenlängen bzw. Periodenlängen der Folgen sind. \\
	Es gilt \( a \mid n\), und daher \(\mu_a \leq \mu_n\) und \(\lambda_a \leq \lambda_n\).
	Die Erwartungswerte für \(\mu_a + \lambda_a\) sowie \(\mu_n + \lambda_n\)
	haben wir bereits im Abschnitt über zufällige Abbildungen betrachtet.
\end{frame}

%------------------------------------------------

\begin{frame}
	\frametitle{Faktorisierung nach Pollard-Rho}
	Um einen Zyklus der Folge \((x_{a_{i+1}})_{i \geq 0}\) zu finden (Floyd oder Brent), obwohl \(a\) unbekannt ist, betrachten wir die Folge
	\((x_{n_{i+1}})_{i \geq 0}\) und prüfen statt der Bedingung \(x_i=x_j\) \(d:=\operatorname{ggT}(x_i-x_j,n)>1\).
	d lässt sich dann wie folgt charakterisieren:
	\[
		d=\prod_{x_i \equiv x_j \pmod p \atop p \mid n,\; p \in \mathbb P} p
	\]
	Im Mittel findet man den ersten Zyklus für den kleinsten Primteiler \(p_1\) von n,
	wenn kein anderer Teiler von \(n\) an jener Stelle einen Zyklus hat, gilt \(d=p_1\).
	Dann haben wir einen Primteiler gefunden und können den Algorithmus mit \(\frac{n}{p_1}\) erneut durchführen.
\end{frame}

%------------------------------------------------

\begin{frame}
	\frametitle{Faktorisierung nach Pollard-Rho}
	Ohne \(x_0\) und/oder \(f\) abzuändern, kann kein Teiler von \(d\) gefunden werden.
	Insbesondere kann der Algorithmus keinen Teiler von \(n\) finden, wenn \(d=n\) gilt.
	Dies definiert man daher als Abbruchbedingung.
	Üblicherweise setzt man \(f(x):=x^2+c\) mit \(c \neq 0, c \neq -2\)
	Da die Berechnung des ggT jedoch verhältnismäßig viel Zeit (log(n) mal Multiplikation) in Anspruch nimmt, berechnet eine optimierte Variante
	\(Q\) (wobei \(log(n) << Q << n^{\frac{1}{4}}\)) mal
	\(q=q \cdot (x_i - x_j)\) und dann \(\operatorname{ggT}(q,d)\) berechnet.
	Dies nimmt in Kauf, das dazwischen zu findende Teiler multipliziert werden.
	Um das zu verhindern kann man den alten Wert von \(x_i\)
	abspeichern und die Suche von dort erneut beginnen für den Fall \(d>1\).
\end{frame}

%------------------------------------------------

\begin{frame}
	\frametitle{Pollard-Rho Faktorisierung}

	Der Algorithmus scheitert genau dann, wenn \( \mu = \mu_p = \mu_n \) und \( \lambda = \lambda_p = \lambda_n \).
	Die Wahrscheinlichkeit dafür ist:
	\begin{align*}
		  \sum_{\lambda=1}^p \sum_{\mu=0}^{p-\lambda} P_p(\mu,\lambda) &\cdot P_n(\mu,\lambda) \\
		= \sum_{\lambda=1}^p \sum_{\mu=0}^{p-\lambda}  \frac{1}{p}\prod_{k=1}^{\mu + \lambda}\left(1-\frac{k}{p}\right) &\cdot
		\frac{1}{n}\prod_{k=1}^{\mu + \lambda}\left(1-\frac{k}{n}\right)\\
		= \frac{1}{n \cdot p} \sum_{\lambda=1}^p \sum_{\mu=0}^{p-\lambda}  \prod_{k=1}^{\mu + \lambda}\left(1-\frac{k}{p}\right) &\cdot
		\left(1-\frac{k}{n}\right)\\
	\end{align*}

\end{frame}

%------------------------------------------------

\begin{frame}
	\frametitle{Pollard-Rho Faktorisierung}

	Der Algorithmus liefert einen Primfaktor \(p_i\) genau dann,
	wenn \( \mu_{p_i} + \lambda_{p_i} < \mu_{p_j} + \lambda_{p_j} \) für \(i \neq j\).
	Die Wahrscheinlichkeit dafür ist:
	\begin{align*}
		  \sum_{\mu_{p_i} + \lambda_{p_i} \leq p_i \atop \mu_{p_i} + \lambda_{p_i} < \mu_{p_j} + \lambda_{p_j} \leq p_j} \prod_{j=1}^M P_{p_j}(\mu_j,\lambda_j) \\
	\end{align*}

\end{frame}

%------------------------------------------------

\begin{frame}
	\frametitle{Laufzeitanalyse der Pollard-Rho Faktorisierung}
	
	Das Pollard-Rho verfahren ist besonders effizient für Zahlen mit einem kleinen Primfaktor.
	Die Zeit um einen Faktor \(p\) zu finden ist im Mittel \(\sqrt{p}\).
	\(\Omega(n)\) ist die Anzahl der (mit Vielfachheit gezählten) Primfaktoren von \(n\).
	Dann gilt: \(\Omega(n) \sim \log(\log(n))\) für \(n \to \infty \) (Satz von Hardy-Ramanujan).
	Es gilt: \(p_1 \leq n^{\frac{1}{\Omega(n)}}\).
	Unter der Annahme, dass die zu faktorisierende Zahl keine Primzahl ist, ist \(\Omega(n) \geq 2\).
	Dann hat man \(p_1 \leq \sqrt{n}\) als exakte Schranke.

\end{frame}

%------------------------------------------------

\begin{frame}
	\frametitle{Referenzen}
		\footnotesize{
			%\begin{thebibliography}{99}
			%	\bibitem[S. Ramanujan; G. H. Hardy, 1917]{p1} S. Ramanujan, G. H. Hardy (1917)
			%	\newblock The normal number of prime factors of a number n
			%	\newblock \emph{Quarterly Journal of Mathematics} 48, 76-92.
			%\end{thebibliography}
			\begin{thebibliography}{99}
				\bibitem[Robert Sedgewick, 1995]{p1} Robert Sedgewick (1995)
				\newblock An Introduction to the Analysis of Algorithms
				\newblock \emph{Addison-Wesley}
			\end{thebibliography}
			\begin{thebibliography}{99}
				\bibitem[J.M. Pollard, 1975]{p1} J.M. Pollard (1975)
				\newblock A Monte Carlo method for factorization
				\newblock \emph{BIT Numerical Mathematics} 15(3), 331-334.
			\end{thebibliography}
			\begin{thebibliography}{99}
				\bibitem[R.P. Brent, 1980]{p1} R.P. Brent (1980)
				\newblock An Improved Monte Carlo Factorization Algorithm
				\newblock \emph{BIT Numerical Mathematics} 20, 176-184.
			\end{thebibliography}
			\begin{thebibliography}{99}
				\bibitem[Donald E. Knuth, 1997]{p1} Donald E. Knuth (1997)
				\newblock The Art of Computer Programming, vol. I: Fundamental Algorithms (3rd ed.)
				\newblock \emph{Addison-Wesley}
			\end{thebibliography}
			\begin{thebibliography}{99}
				\bibitem[Donald E. Knuth, 1997]{p1} Donald E. Knuth (1997)
				\newblock The Art of Computer Programming, vol. II: Seminumerical Algorithms (3rd ed.)
				\newblock \emph{Addison-Wesley}
			\end{thebibliography}
		}
\end{frame}

%------------------------------------------------

%----------------------------------------------------------------------------------------

\end{document}
